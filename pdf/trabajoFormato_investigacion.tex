\documentclass[11pt,openbib, letter]{article}
\oddsidemargin=1cm
% Incluir los paquetes necesarios 
\usepackage[latin1]{inputenc}%paquete principal2
\usepackage{latexsym} % Simbolos 
\usepackage{graphicx} % Inclusion de graficos. Soporte para figura 
\usepackage{hyperref} % Soporte hipertexto
\usepackage[spanish]{babel}
\usepackage{epsfig}
\usepackage{graphicx}
\usepackage{rotating}
\usepackage{epstopdf}
\usepackage{longtable}
\usepackage{graphicx}
\usepackage[centertags]{amsmath}
\usepackage{amsthm}
\usepackage{array}
\usepackage{times}
\usepackage[left=1in, right=1in, top=1in, bottom=1in]{geometry}
\usepackage{rotating}
\usepackage{amstext}
\usepackage{pdfpages}
\title{\textbf{Contenido m�nimo para el Articulo de Investigaci�n} \\ Materia: Miner�a de Datos 2}
\author{Docente: MSc. Alvaro Limber Chirino Gutierrez}
\date{}
\begin{document} % Inicio del documento
%\renewcommand{\abstractname}{Resumen}
%\renewcommand{\contentsname}{Contenido} 
%\renewcommand{\appendixname}{Apéndice} 
%\renewcommand{\refname}{Referencias} 
%\renewcommand{\figurename}{Figura} 
%\renewcommand{\listfigurename}{�?ndice de figuras} 
%\renewcommand{\tablename}{Tabla}
\maketitle
%%%%%%%%%%%%%%%%%%%%%%%%%%%%%%%%%%%%%%%%%%%%%%%
%%%%%%%%%%%%%%%%%%%%%%%%%%%%%%%%%%%%%%%%%%%%%%%
\section{Caracter�sticas}
\begin{itemize}
\item Tipo de trabajo: Articulo de investigaci�n aplicando m�todos de miner�a de datos 
\begin{itemize}
\item El proyecto es individual
\item Debe trabajar sobre una base de datos nacional con al menos 10 mil observaciones
\item La tem�tica de la base de datos es a elecci�n del estudiante
\item Se debe incluir m�todos vistos en la materia
\end{itemize}
\item Los trabajos no deben exceder las 10 p�ginas (sin contar anexos y caratula)
\item El art�culo debe escribirse en Rmarkdown usando la plantilla SAGE del paquete rticles
\item El trabajo debe contar con un repositorio en GitHub
\item Entregables en repositorio de GitHub:
\begin{itemize}
\item C�digo de R
\item Articulo de investigaci�n
\item Presentaci�n
\end{itemize}
\end{itemize}

\section{Contenido m�nimo del articulo de trabajo}
El contenido m�nimo es:
\begin{itemize}
\item \textbf{Introducci�n}
\item \textbf{Objetivos}
\item \textbf{Motivaci�n}
\item \textbf{Marco Te�rico / Revisi�n de literatura}
\item \textbf{Descripci�n del dataset}
\item \textbf{Metodolog�a}
\item \textbf{Resultados y an�lisis}
\item \textbf{Conclusiones y recomendaciones}
\item \textbf{Referencias} En formato .bib
\end{itemize}

\section{Fechas importantes}

Se destina una sesi�n en la semana correspondiente para revisar avances.

\begin{itemize}
\item Semana 6: Motivaci�n, marco te�rico y posible dataset
\item Semana 13: Metodolog�a y resultados preliminares
\item Semana 18: Seguimiento general
\item Semana 19: Presentaci�n y Entrega de trabajo
\end{itemize}

\end{document}